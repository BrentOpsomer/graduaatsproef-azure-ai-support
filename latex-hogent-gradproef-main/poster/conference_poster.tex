%==============================================================================
% Poster graduaatsproef
%==============================================================================

\documentclass[a0,portrait]{hogent-poster}

% Info over de opleiding
\course{Graduaatsproef}
\studyprogramme{Graduaat in het Programmeren}
\academicyear{2025-2026}
\institution{Hogeschool Gent, Valentin Vaerwyckweg 1, 9000 Gent}

% Info over de graduaatsproef
\title{Automatische samenvatting van support-e-mails met Azure AI}
\subtitle{Proof of Concept voor toekomstige integratie in een supportomgeving}
\author{Brent Opsomer}
\email{brent.opsomer@student.hogent.be}
\supervisor{Luc Vervoort}
\cosupervisor{Kris Van Achter (bedrijf)}

\specialisation{Programmeren}
\keywords{Azure OpenAI, GPT-4o-mini, Support, Automatisatie}
% \projectrepo{https://github.com/BrentOpsomer/graduaatsproef-azure-ai-support}

\begin{document}

\maketitle

\begin{abstract}
Supportmedewerkers verwerken dagelijks tickets die vaak bestaan uit lange en ongestructureerde e-mailconversaties. Het manueel doornemen en interpreteren van deze informatie is tijdrovend en kan leiden tot inconsistente documentatie. In deze graduaatsproef wordt een proof of concept uitgewerkt die gebruikmaakt van Azure OpenAI om automatisch kernachtige samenvattingen te genereren van support-e-mails. De oplossing werd ontwikkeld als een afzonderlijke demo-applicatie en toont aan hoe artificiële intelligentie kan bijdragen tot efficiëntere ticketafhandeling en betere kennisopbouw binnen supportteams.
\end{abstract}

\begin{multicols}{2}

\section{Introductie}
Binnen moderne IT-supportomgevingen worden e-mails vaak rechtstreeks omgezet in tickets. Deze berichten bevatten echter niet altijd gestructureerde informatie en zijn soms uitgebreid of herhalend. Hierdoor gaat waardevolle tijd verloren aan het interpreteren en samenvatten van de kern van het probleem. Een geautomatiseerde aanpak kan deze administratieve last aanzienlijk verminderen en de kwaliteit van ticketdocumentatie verhogen.

\section{Doelstelling}
Het doel van deze graduaatsproef is het ontwikkelen van een proof of concept die aantoont hoe AI-technologie ingezet kan worden om support-e-mails automatisch samen te vatten. De focus ligt op het creëren van een realistische en uitbreidbare oplossing die later eenvoudig geïntegreerd kan worden in een bestaande supporttool.

\section{Aanpak}
De proof of concept bestaat uit een demo-applicatie waarbij de gebruiker een support-e-mail selecteert of invoert. Deze e-mail wordt doorgestuurd naar een .NET-backend die communiceert met Azure OpenAI. Het gekozen model, \textbf{GPT-4o-mini}, genereert vervolgens een beknopte en professionele samenvatting. Deze wordt in de gebruikersinterface weergegeven en kan in een latere fase automatisch aan een ticket worden toegevoegd.

\section{Architectuur}
De architectuur is bewust modulair opgezet en bestaat uit drie componenten: een React-frontend, een .NET minimal API-backend en de Azure OpenAI Service. De backend fungeert als centrale schakel en verzorgt de beveiliging, validatie en communicatie met het AI-model. Deze opzet maakt het mogelijk om de oplossing later zonder grote aanpassingen te integreren in bestaande systemen.

\section{Resultaten}
De proof of concept toont aan dat Azure OpenAI in staat is om lange support-e-mails snel en accuraat samen te vatten. De gegenereerde samenvattingen zijn duidelijk, consistent en direct bruikbaar in een ticketsysteem. Dit leidt tot een betere leesbaarheid van tickets en ondersteunt snellere probleemanalyse door supportmedewerkers.

\section{Conclusie}
Deze graduaatsproef bevestigt dat artificiële intelligentie een meerwaarde kan bieden binnen IT-supportprocessen. Door het automatiseren van samenvattingen wordt niet alleen tijd gewonnen, maar wordt ook de kwaliteit van documentatie verhoogd. De ontwikkelde proof of concept vormt een solide basis voor verdere implementatie in een productieomgeving.

\section{Toekomstig werk}
Mogelijke uitbreidingen omvatten de integratie in een bestaande supporttool, automatische classificatie van tickets, sentimentanalyse en koppeling met een knowledge base. Daarnaast kunnen extra veiligheidsmaatregelen worden toegevoegd, zoals PII-detectie, logging en strengere contentfilters.

\end{multicols}
\end{document}
