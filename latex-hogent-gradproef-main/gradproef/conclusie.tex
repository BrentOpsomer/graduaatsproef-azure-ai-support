%%=============================================================================
%% Conclusie
%%=============================================================================

\chapter{Conclusie}%
\label{ch:conclusie}

Deze graduaatsproef had als doel te onderzoeken hoe automatische samenvatting van e-mails met behulp van artificiële intelligentie geïntegreerd kan worden in een ticketingsysteem en welke meerwaarde dit biedt op vlak van efficiëntie, kwaliteit en kost. Aan de hand van een praktijkgerichte aanpak werd een werkende proof-of-concept ontwikkeld die deze integratie in een realistische context demonstreert.

De centrale onderzoeksvraag kan positief worden beantwoord. Het is technisch haalbaar om e-mails automatisch te laten samenvatten via een cloudgebaseerd AI-model en deze samenvatting toe te voegen aan tickets. Door gebruik te maken van het GPT-4o-mini model via Azure AI Foundry werd aangetoond dat een evenwicht kan worden gevonden tussen kwaliteit van de samenvattingen en beheersbare kosten. De integratie met een backend Web API en een eenvoudige frontend testomgeving bevestigt dat de oplossing schaalbaar en flexibel inzetbaar is.

Op het vlak van efficiëntie biedt de oplossing een duidelijke meerwaarde. Supportmedewerkers hoeven niet langer volledige e-mailconversaties door te nemen om de kern van een probleem te begrijpen. De automatisch gegenereerde samenvatting geeft meteen inzicht in de context van het ticket, wat de analysetijd verkort en de doorlooptijd van tickets kan verminderen. Dit draagt bij tot een vlottere werking van de servicedesk en een hogere productiviteit.

Ook de kwaliteit van de ticketdocumentatie wordt verbeterd. Waar tickets voorheen vaak bestonden uit lange en ongestructureerde e-mailthreads, zorgen de samenvattingen voor duidelijke en beknopte probleemomschrijvingen. Dit komt niet alleen de directe opvolging ten goede, maar verhoogt ook de bruikbaarheid van tickets als bron voor kennisbeheer en rapportering.

De kostenanalyse toont aan dat het gebruik van een AI-model voor deze toepassing financieel haalbaar is. Zelfs bij een relatief hoog volume van samenvattingen per dag blijft de jaarlijkse kost beperkt. Dit maakt de oplossing niet alleen technisch, maar ook economisch interessant voor organisaties die hun ticketafhandeling willen optimaliseren zonder grote investeringen te moeten doen.

Toch zijn er ook enkele aandachtspunten en beperkingen. De kwaliteit van de samenvattingen blijft afhankelijk van de invoer: onduidelijke of onvolledige e-mails kunnen leiden tot minder accurate samenvattingen. Daarnaast moet er blijvend aandacht zijn voor dataveiligheid en privacy, zeker wanneer gevoelige informatie via externe AI-diensten wordt verwerkt. In een productieomgeving is het aangewezen om bijkomende maatregelen te nemen, zoals logging, monitoring en duidelijke afspraken rond dataverwerking.

Deze graduaatsproef roept bovendien nieuwe vragen op die uitnodigen tot verder onderzoek. Zo kan onderzocht worden in welke mate automatische samenvatting gecombineerd kan worden met andere AI-functionaliteiten, zoals automatische classificatie van tickets of het voorstellen van oplossingsrichtingen. Ook een uitgebreidere gebruikerstest met meerdere supportmedewerkers zou waardevolle inzichten kunnen opleveren over de impact op de dagelijkse werking en de acceptatie van dergelijke technologieën.

Samenvattend kan gesteld worden dat deze graduaatsproef aantoont dat automatische e-mailsamenvatting met behulp van artificiële intelligentie een haalbare en waardevolle aanvulling is voor moderne ticketingsystemen. De gerealiseerde proof-of-concept biedt een stevige basis voor verdere ontwikkeling en mogelijke implementatie in een productieomgeving, met duidelijke voordelen voor zowel supportmedewerkers als organisaties in het algemeen.
