\chapter{Conclusie}%
\label{ch:conclusie}

Deze graduaatsproef had als doel te onderzoeken hoe automatische samenvatting van e-mails met behulp van artificiële intelligentie technisch gerealiseerd kan worden en welke meerwaarde dit kan bieden binnen de context van IT-support en ticketverwerking. Aan de hand van een praktijkgerichte aanpak werd een werkende proof-of-concept ontwikkeld die aantoont hoe e-mailinhoud automatisch kan worden samengevat met behulp van een cloudgebaseerd AI-model.

De centrale onderzoeksvraag kan positief worden beantwoord. Het is technisch haalbaar om e-mails automatisch te laten samenvatten via een AI-model zoals GPT-4o-mini, aangeboden via Azure AI Foundry. De ontwikkelde proof-of-concept toont aan dat met een backend Web API en een eenvoudige frontend op een betrouwbare en schaalbare manier samenvattingen gegenereerd kunnen worden, met beheersbare kosten.

Op het vlak van efficiëntie biedt deze aanpak een duidelijke potentiële meerwaarde voor supportomgevingen. Door lange en ongestructureerde e-mailconversaties automatisch samen te vatten, kan sneller inzicht worden verkregen in de kern van een probleem. Hoewel de proof-of-concept losstaat van een concreet ticketsysteem, tonen de resultaten aan dat dergelijke samenvattingen supportmedewerkers kunnen ondersteunen bij analyse en opvolging.

Ook op het vlak van documentatie biedt automatische e-mailsamenvatting voordelen. Beknopte en consistente samenvattingen verhogen de leesbaarheid van informatie en vormen een betere basis voor kennisopbouw en hergebruik. De proof-of-concept illustreert hoe deze samenvattingen gegenereerd kunnen worden, zonder reeds een directe koppeling met bestaande tickettools te vereisen.

De kostenanalyse wijst uit dat het gebruik van een cloudgebaseerd AI-model voor deze toepassing financieel haalbaar blijft, zelfs bij een relatief hoog aantal samenvattingen per dag. Hierdoor vormt de oplossing niet alleen een technisch, maar ook een economisch verantwoorde basis voor verdere ontwikkeling.

\begin{table}[h]
\centering
\caption{Geschatte maandelijkse kost voor automatische e-mailsamenvatting met GPT-4o-mini}
\label{tab:monthlycost}
\begin{tabular}{r r r}
\hline
Gemiddeld per dag & Per maand & Kost per maand (\$) \\
\hline
10   & 300   & 0.17 \\
50   & 1\,500 & 0.83 \\
100  & 3\,000 & 1.67 \\
250  & 7\,500 & 4.16 \\
500  & 15\,000 & 8.33 \\
1\,000 & 30\,000 & 16.65 \\
\hline
\end{tabular}
\end{table}

Tegelijk zijn er enkele aandachtspunten en beperkingen. De kwaliteit van de gegenereerde samenvattingen blijft afhankelijk van de aangeleverde e-mailinhoud. Daarnaast is dataveiligheid en privacy een belangrijk aandachtspunt bij het verwerken van e-mails via externe AI-diensten. In een productiecontext zijn bijkomende maatregelen zoals logging, monitoring en duidelijke afspraken rond gegevensverwerking noodzakelijk.

Deze graduaatsproef biedt een basis voor verder onderzoek en uitbreiding. Mogelijke vervolgstappen zijn de integratie van de samenvattingsfunctionaliteit in een bestaand ticketingsysteem, gecombineerd met aanvullende AI-functionaliteiten zoals automatische ticketclassificatie of sentimentanalyse. Ook uitgebreidere gebruikerstesten kunnen bijkomende inzichten opleveren over de impact op de dagelijkse werking van supportteams.

Samenvattend kan gesteld worden dat deze graduaatsproef aantoont dat automatische samenvatting van e-mails met behulp van artificiële intelligentie technisch haalbaar is en een duidelijke meerwaarde kan bieden voor IT-supportomgevingen. De gerealiseerde proof-of-concept vormt een solide en realistische basis voor verdere ontwikkeling en mogelijke integratie in productieomgevingen.
