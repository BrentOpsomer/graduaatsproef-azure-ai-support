\chapter{\IfLanguageName{dutch}{Stand van zaken}{State of the art}}%
\label{ch:standvanzaken}

Dit hoofdstuk bouwt verder op de inleiding en plaatst de graduaatsproef in een bredere context. Waar in het vorige hoofdstuk het probleem en de doelstelling werden geschetst, focust dit hoofdstuk op de bestaande kennis en ontwikkelingen binnen het domein van ticketing, e-mailverwerking en automatische tekstsamenvatting met behulp van artificiële intelligentie. Deze stand van zaken vormt de theoretische basis voor de keuzes die in het volgende hoofdstuk, de methodologie, worden toegelicht.

\section{Ticketing in IT-ondersteuning}

Ticketsystemen zijn een essentieel onderdeel van moderne IT-ondersteuning. Ze worden gebruikt om incidenten, service requests en veranderingen te registreren en op te volgen. Naast hun operationele functie spelen tickets ook een belangrijke rol in rapportering, SLA-opvolging en kennisbeheer . 

In de praktijk blijkt echter dat de kwaliteit van ticketinformatie sterk varieert. Wanneer tickets worden aangemaakt op basis van e-mailverkeer, wordt de inhoud vaak letterlijk overgenomen of als bijlage toegevoegd. Dit leidt tot lange, ongestructureerde ticketbeschrijvingen die moeilijk leesbaar zijn voor andere medewerkers. Onderzoek toont aan dat onduidelijke ticketinformatie een negatieve impact heeft op zowel de oplostijd als de klanttevredenheid \autocite{Creeger2009}.

\section{E-mail als communicatiekanaal}

Ondanks de opkomst van chat- en portaaloplossingen blijft e-mail één van de belangrijkste communicatiekanalen tussen eindgebruikers en IT-servicedesks. Gebruikers beschrijven hun probleem vaak in meerdere berichten, soms verspreid over meerdere dagen. Hierdoor ontstaat een versnipperde informatiestroom die voor supportmedewerkers moeilijk te overzien is.

Het manueel doornemen en interpreteren van deze communicatie vraagt tijd en concentratie. Bovendien is de kwaliteit van handmatige samenvattingen sterk afhankelijk van de ervaring en werkdruk van de medewerker. Dit kan leiden tot inconsistente documentatie, wat nadelig is voor samenwerking binnen het team en voor latere hergebruik in een knowledge base.

\section{Automatische tekstsamenvatting}

Automatische tekstsamenvatting is een belangrijk onderzoeksdomein binnen de natuurlijke taalverwerking. Traditioneel wordt een onderscheid gemaakt tussen extractieve en abstractieve samenvatting. Bij extractieve samenvatting worden belangrijke zinnen letterlijk uit de brontekst geselecteerd, terwijl abstractieve samenvatting nieuwe zinnen genereert die de kern van de tekst weergeven.

Recente ontwikkelingen in deep learning, en meer specifiek in transformer-gebaseerde modellen, hebben geleid tot een sterke vooruitgang in abstractieve samenvatting. Deze modellen zijn in staat om de semantische betekenis van teksten te begrijpen en deze om te zetten in coherente en beknopte samenvattingen. Hierdoor zijn ze bijzonder geschikt voor toepassingen zoals e-mailsamenvatting in professionele contexten.

\section{AI-modellen in de cloud}

Cloudplatformen zoals Microsoft Azure bieden organisaties de mogelijkheid om geavanceerde AI-modellen te gebruiken zonder zelf complexe infrastructuur te moeten beheren. Via diensten zoals Azure AI Foundry kunnen modellen eenvoudig worden gedeployed en aangesproken via API’s. Dit verlaagt de drempel voor integratie van artificiële intelligentie in bestaande bedrijfsprocessen.

Een belangrijk voordeel van cloudgebaseerde AI is de schaalbaarheid. Organisaties kunnen het gebruik van AI aanpassen aan hun noden, zonder grote voorafgaande investeringen in hardware. Tegelijk brengt dit nieuwe aandachtspunten met zich mee, zoals dataveiligheid, privacy en kostenbeheersing. Deze aspecten zijn bijzonder relevant wanneer gevoelige informatie, zoals e-mailinhoud, verwerkt wordt.

\section{AI in servicedesk-omgevingen}

De toepassing van artificiële intelligentie in servicedesk-omgevingen is de afgelopen jaren sterk toegenomen. Voorbeelden zijn chatbots voor first-line support, automatische classificatie van tickets en voorspellende analyses voor incidentbeheer. Automatische samenvatting van e-mails sluit logisch aan bij deze evolutie en vormt een volgende stap in de automatisering van ondersteunende processen.

Onderzoek wijst uit dat dergelijke toepassingen niet bedoeld zijn om medewerkers te vervangen, maar om hen te ondersteunen door repetitieve taken te automatiseren \autocite{Creeger2009}. Hierdoor kunnen supportmedewerkers zich focussen op complexere problemen en klantgerichte taken, wat de algemene kwaliteit van dienstverlening verhoogt.

\section{Positionering van deze graduaatsproef}

Binnen dit bredere kader positioneert deze graduaatsproef zich op het snijvlak van ticketing, e-mailverwerking en cloudgebaseerde AI. Door gebruik te maken van het GPT-4o-mini model via Azure AI Foundry wordt onderzocht hoe automatische samenvatting van e-mailinhoud technisch gerealiseerd kan worden binnen een context die aansluit bij bestaande ticketflows.

In tegenstelling tot louter theoretisch onderzoek focust deze graduaatsproef op een concrete proof-of-concept in een realistische context. De literatuurstudie toont aan dat de gebruikte technologieën voldoende matuur zijn om ingezet te worden in productieomgevingen. Dit vormt de basis voor de methodologische keuzes die in het volgende hoofdstuk worden besproken.

