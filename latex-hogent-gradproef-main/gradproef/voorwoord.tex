\chapter*{\IfLanguageName{dutch}{Woord vooraf}{Preface}}%
\label{ch:voorwoord}

Deze graduaatsproef is tot stand gekomen binnen de context van ticket- en
incidentopvolging, waar snelheid en correcte documentatie een grote impact
hebben op de kwaliteit van support. Tijdens mijn stage merkte ik dat tickets
vaak informatie bevatten die verspreid zit over e-mailverkeer. Het manueel doornemen, 
samenvatten of kopiëren van relevante stukken kost tijd en zorgt er
ook voor dat belangrijke context soms niet, of onvolledig, in het ticket terechtkomt.


Daarom koos ik voor een oplossing die e-mails, gekoppeld aan nieuwe tickets
binnen de bestaande ticketflow, automatisch laat samenvatten met behulp van
een AI-model via Microsoft Azure. Bij het uitwerken van deze graduaatsproef
werd rekening gehouden met de bestaande codebase en architectuur van het
stagebedrijf. De oplossing zelf werd ontwikkeld als een afzonderlijke
proof-of-concept, los van de productieomgeving, maar afgestemd op een
realistische bedrijfscontext. De gegenereerde samenvatting wordt toegevoegd
aan het ticket binnen de demo-oplossing.

Het doel van deze graduaatsproef is om tickets sneller begrijpelijk te maken
voor collega’s, de oplostijd te verkorten en tegelijk de kwaliteit van de
ticketdocumentatie te verbeteren. Daarnaast biedt deze aanpak ook meerwaarde
voor een knowledge base, omdat de kern van het probleem en de bijhorende
context beter gestructureerd beschikbaar worden.

Ik wil graag mijn promotor en begeleiders bedanken voor de feedback en de
richting die zij gaven tijdens het uitwerken van deze opdracht. Ook bedank
ik mijn stagebedrijf voor de kans om dit onderwerp in een realistische
omgeving uit te werken, inclusief de mogelijkheid om de volledige flow te
testen, van e-mailverwerking tot backend en frontend, en de validatie van de
oplossing met tools zoals Postman. Tot slot wil ik iedereen bedanken die op
enige wijze heeft bijgedragen aan het tot stand komen van deze graduaatsproef.
