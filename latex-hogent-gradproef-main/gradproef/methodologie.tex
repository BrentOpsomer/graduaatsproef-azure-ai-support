%%=============================================================================
%% Methodologie
%%=============================================================================

\chapter{\IfLanguageName{dutch}{Methodologie}{Methodology}}%
\label{ch:methodologie}

In dit hoofdstuk wordt toegelicht welke aanpak werd gevolgd bij het uitvoeren van deze graduaatsproef. De focus ligt op een praktijkgerichte methodologie, waarbij een concrete technische proof-of-concept werd uitgewerkt, getest en geëvalueerd. Dit hoofdstuk beschrijft niet de volledige technische implementatie in detail, maar geeft een overzicht van de verschillende fasen en de bijhorende doelstellingen.

\section{Onderzoeksstrategie}

De graduaatsproef volgt een ontwerpgerichte onderzoeksstrategie. Dit betekent dat het onderzoek niet louter beschrijvend is, maar gericht op het ontwikkelen van een werkbare oplossing voor een concreet probleem uit de praktijk. De centrale onderzoeksvraag wordt benaderd door het realiseren van een proof-of-concept en het evalueren van de werking ervan binnen een realistische supportcontext.

\section{Fasering van het onderzoek}

Het onderzoek werd opgedeeld in zes opeenvolgende fasen. Elke fase heeft een duidelijke doelstelling en bijhorende output.

\subsection{Analysefase}

In deze fase werd de context van ticketgebaseerde supportomgevingen geanalyseerd, met bijzondere aandacht voor de manier waarop e-mails doorgaans worden gebruikt bij het aanmaken en opvolgen van tickets. Het doel was om de belangrijkste knelpunten te identificeren, zoals het gebrek aan structuur in e-mailinhoud en de tijd die nodig is om e-mailverkeer manueel te interpreteren.

\textbf{Deliverable:} overzicht van het probleemdomein en afbakening van de scope.

\subsection{Literatuur- en technologieverkenning}

Vervolgens werd nagegaan welke bestaande technologieën beschikbaar zijn voor automatische tekstsamenvatting. Hierbij werd zowel gekeken naar academische inzichten als naar praktische cloudoplossingen. Op basis hiervan werd gekozen voor het gebruik van het GPT-4o-mini model via Azure AI Foundry, omwille van de balans tussen performantie en kostprijs.

\textbf{Deliverable:} onderbouwde keuze van technologie en platform.

\subsection{Ontwerpfase}

In de ontwerpfase werd de architectuur van de proof-of-concept uitgewerkt. Hierbij werd bepaald hoe de verschillende componenten met elkaar communiceren: de backend Web API, het AI-model binnen Azure en een eenvoudige frontend demo-interface. Daarnaast werden aspecten zoals configuratiebeheer en beveiliging van API-sleutels vastgelegd.

\begin{figure}[h]
\centering
\includegraphics[width=0.9\textwidth]{images/architectuur.png}
\caption{Architectuuroverzicht van de proof-of-concept}
\label{fig:architecture}
\end{figure}

\textbf{Deliverable:} technisch ontwerp van de proof-of-concept en bijhorende datastromen.

\subsection{Implementatiefase}

Tijdens de implementatiefase werd het ontwerp omgezet in een werkende proof-of-concept. Er werd een backend Web API opgezet die e-mailinhoud ontvangt, verwerkt en doorstuurt naar het AI-model. De configuratie werd veilig beheerd via user secrets. Daarnaast werd een eenvoudige frontend ontwikkeld met React (Vite) om de volledige flow, van e-mailinvoer tot gegenereerde samenvatting, te kunnen demonstreren.

\begin{listing}[h]
\caption{Backend-aanroep naar Azure OpenAI voor het genereren van een samenvatting}
\label{lst:openai}
\begin{minted}{csharp}
public async Task<string> SummarizeAsync(HttpClient http, string text)
{
    var payload = JsonSerializer.Serialize(new
    {
        messages = new[]
        {
            new { role = "system", content = SystemPrompt },
            new { role = "user", content = text },
        },
        max_tokens = 200,
        temperature = 0.2
    });

    var req = new HttpRequestMessage(HttpMethod.Post, Url);
    req.Headers.Add("api-key", _apiKey);
    req.Content = new StringContent(payload, Encoding.UTF8, "application/json");

    var resp = await http.SendAsync(req);
    resp.EnsureSuccessStatusCode();

    using var doc = JsonDocument.Parse(await resp.Content.ReadAsStringAsync());
    return doc.RootElement
        .GetProperty("choices")[0]
        .GetProperty("message")
        .GetProperty("content")
        .GetString();
}
\end{minted}
\end{listing}

\begin{figure}[h]
\centering

\includegraphics[width=0.8\textwidth]{images/supportui_1.png}

\vspace{0.5em}

\includegraphics[width=0.8\textwidth]{images/supportui_2.png}

\vspace{0.5em}

\includegraphics[width=0.8\textwidth]{images/supportui_3.png}

\caption[Demo-flow]{Demo-flow van e-mailupload tot AI-samenvatting}
\label{fig:demoflow}

\end{figure}


\textbf{Deliverable:} werkende proof-of-concept met volledige end-to-end flow.

\subsection{Testfase}

De oplossing werd handmatig getest met behulp van Postman en verschillende praktijkscenario’s. Daarbij werd nagegaan of e-mails correct werden verwerkt en of het AI-model een bruikbare samenvatting terugstuurde. De gegenereerde samenvattingen werden manueel gecontroleerd om te beoordelen of de context van de e-mailconversaties correct werd weergegeven en of de communicatie tussen backend en AI-model stabiel verliep.
\textbf{Deliverable:} testresultaten en evaluatie van de werking.

\subsection{Evaluatie- en rapportagefase}

Tot slot werden de resultaten geanalyseerd en werd een kostenraming opgesteld op basis van het verwachte aantal samenvattingen per dag. De bevindingen werden verwerkt in dit rapport, samen met conclusies en aanbevelingen voor verdere optimalisatie en toekomstig gebruik.

\textbf{Deliverable:} eindrapport met conclusie en toekomstperspectief.

\section{Verantwoording van de aanpak}

De gekozen aanpak sluit nauw aan bij het karakter van een graduaatsproef in het domein Programmeren. Door te werken met een concrete case en een tastbaar proof-of-concept wordt niet alleen de technische haalbaarheid onderzocht, maar ook de potentiële meerwaarde voor een realistische werkomgeving. Deze methodologie laat toe om de onderzoeksvraag niet alleen theoretisch, maar ook empirisch te benaderen aan de hand van een werkende demo-oplossing.
