%%=============================================================================
%% Management summaries + academische samenvatting
%%=============================================================================

\chapter*{Management summary}

Deze graduaatsproef onderzoekt hoe automatische samenvatting van support-e-mails met behulp van artificiële intelligentie kan worden ingezet om IT-supportprocessen efficiënter en consistenter te maken. In veel servicedesk-omgevingen worden tickets aangemaakt op basis van lange en ongestructureerde e-mailconversaties, wat leidt tot tijdverlies, interpretatiefouten en onvolledige documentatie.

Binnen deze opdracht werd een proof-of-concept ontwikkeld dat inkomende e-mails automatisch laat samenvatten via het GPT-4o-mini taalmodel, aangeboden via Azure AI Foundry. De oplossing bestaat uit een .NET backend die e-mailinhoud verwerkt en doorstuurt naar het AI-model, en een React-gebaseerde frontend die de gegenereerde samenvatting visualiseert voor de gebruiker.

De oplossing toont aan dat automatische e-mailsamenvatting technisch betrouwbaar en economisch haalbaar is. Supportmedewerkers krijgen sneller inzicht in de kern van een probleem, waardoor tickets efficiënter kunnen worden opgevolgd en beter bruikbaar worden voor kennisopbouw. De maandelijkse kosten blijven laag, zelfs bij grotere volumes, wat de technologie geschikt maakt voor productiegebruik.

Deze proof-of-concept vormt een realistische basis voor verdere integratie in bestaande ticketsystemen en voor uitbreiding met bijkomende AI-functionaliteiten zoals ticketclassificatie en sentimentanalyse.

\clearpage

%%=============================================================================
%% Management summary (English)
%%=============================================================================
\chapter*{Management summary (English)}

This thesis investigates how automatic summarization of support emails using artificial intelligence can improve efficiency and documentation quality in IT support environments. In many service desks, tickets are created based on long and unstructured email conversations, which leads to information overload, misinterpretation, and inconsistent documentation.

A proof-of-concept was developed using the GPT-4o-mini language model via Azure AI Foundry. A .NET backend processes incoming email content and sends it to the AI model, while a React-based frontend displays the generated summaries to support agents.

The results show that automated email summarization is technically reliable and economically viable. Support staff gain faster insight into the core of each case, which improves ticket handling, reduces resolution time, and strengthens knowledge management. Even at higher usage volumes, the operational costs remain low.

This proof-of-concept provides a solid foundation for future integration into existing ticketing systems and for further AI-driven extensions such as ticket classification and sentiment analysis.

\clearpage


%%=============================================================================
%% Academische samenvatting
%%=============================================================================

\IfLanguageName{english}{%
\selectlanguage{dutch}
\chapter*{Samenvatting}

\selectlanguage{english}
}{}

%% Hoofdsamenvatting (taal van het document)
\chapter*{\IfLanguageName{dutch}{Samenvatting}{Abstract}}

Deze graduaatsproef onderzoekt hoe automatische samenvatting van e-mails met behulp van artificiële intelligentie technisch gerealiseerd kan worden en welke meerwaarde dit kan bieden binnen IT-supportomgevingen. In veel servicedesk-omgevingen wordt e-mailverkeer gebruikt als basis voor verdere opvolging, maar deze communicatie is vaak lang, ongestructureerd en moeilijk snel te interpreteren.

Het doel van deze graduaatsproef is na te gaan hoe e-mails automatisch kunnen worden samengevat aan de hand van een AI-gestuurde oplossing en hoe deze samenvattingen kunnen worden weergegeven in een proof-of-concept ter ondersteuning van supportprocessen. De centrale onderzoeksvraag focust op de technische haalbaarheid, efficiëntie en kosteneffectiviteit van automatische e-mailsamenvatting met behulp van cloudgebaseerde AI.

De gekozen methodologie is praktijkgericht en focust op het bouwen van een proof-of-concept. Hiervoor werd het GPT-4o-mini model via Azure AI Foundry gebruikt. Er werd een backend Web API ontwikkeld om e-mailinhoud te verwerken en door te sturen naar het AI-model, waarbij gevoelige configuratiegegevens, zoals API-sleutels, via user secrets werden beheerd. Daarnaast werd een eenvoudige frontend opgezet met React (Vite) om de volledige flow van e-mailinvoer tot gegenereerde samenvatting te demonstreren. De API werd eerst handmatig getest met Postman, waarna de volledige applicatie werd uitgeprobeerd met realistische e-mails om te controleren of het systeem correct functioneerde.
De resultaten tonen aan dat automatische e-mailsamenvatting technisch haalbaar is en consistente, bruikbare samenvattingen oplevert. Deze samenvattingen maken het mogelijk om sneller inzicht te krijgen in de kern van e-mailconversaties, wat een duidelijke potentiële meerwaarde biedt voor supportmedewerkers en ondersteunende processen.

De kostenanalyse wijst uit dat het gebruik van een cloudgebaseerd AI-model voor deze toepassing financieel haalbaar blijft, zelfs bij een relatief hoog aantal samenvattingen per dag. Hierdoor vormt de oplossing niet alleen een technisch, maar ook een economisch verantwoorde basis voor verdere ontwikkeling.

Samenvattend toont deze graduaatsproef aan dat automatische samenvatting van e-mails met behulp van artificiële intelligentie een haalbare en waardevolle technologie is binnen IT-supportcontexten. De gerealiseerde proof-of-concept vormt een solide uitgangspunt voor toekomstige integratie in bestaande systemen en verdere uitbreiding met bijkomende AI-functionaliteiten.
