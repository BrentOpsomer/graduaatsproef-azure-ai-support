%%=============================================================================
%% Samenvatting
%%=============================================================================

%%---------- Nederlandse samenvatting -----------------------------------------

\IfLanguageName{english}{%
\selectlanguage{dutch}
\chapter*{Samenvatting}

Deze graduaatsproef onderzoekt hoe automatische samenvatting van e-mails met behulp van artificiële intelligentie technisch gerealiseerd kan worden en welke meerwaarde dit kan bieden binnen IT-supportomgevingen. In veel servicedesk-omgevingen wordt e-mailverkeer gebruikt als basis voor verdere opvolging, maar deze communicatie is vaak lang, ongestructureerd en moeilijk snel te interpreteren.

Het doel van deze graduaatsproef is na te gaan hoe e-mails automatisch kunnen worden samengevat aan de hand van een AI-gestuurde oplossing en hoe deze samenvattingen kunnen worden weergegeven in een proof-of-concept ter ondersteuning van supportprocessen. De centrale onderzoeksvraag focust op de technische haalbaarheid, efficiëntie en kosteneffectiviteit van automatische e-mailsamenvatting met behulp van cloudgebaseerde AI.

De gekozen methodologie is praktijkgericht en richt zich op het ontwikkelen van een proof-of-concept. Hiervoor werd gebruikgemaakt van het GPT-4o-mini model via Azure AI Foundry. Een backend Web API werd ontwikkeld om e-mailinhoud te verwerken en door te sturen naar het AI-model. Gevoelige configuratiegegevens, zoals API-sleutels, worden veilig beheerd via user secrets. Daarnaast werd een eenvoudige frontend opgezet met React (Vite) om de volledige flow van e-mailinvoer tot gegenereerde samenvatting te demonstreren en te testen. De oplossing werd gevalideerd met behulp van Postman en end-to-end tests.

De resultaten tonen aan dat automatische e-mailsamenvatting technisch haalbaar is en consistente, bruikbare samenvattingen oplevert. Deze samenvattingen maken het mogelijk om sneller inzicht te krijgen in de kern van e-mailconversaties, wat een duidelijke potentiële meerwaarde biedt voor supportmedewerkers en ondersteunende processen.

De kostenanalyse wijst uit dat het gebruik van een cloudgebaseerd AI-model voor deze toepassing financieel haalbaar blijft, zelfs bij een relatief hoog aantal samenvattingen per dag. Hierdoor vormt de oplossing niet alleen een technisch, maar ook een economisch verantwoorde basis voor verdere ontwikkeling.

Samenvattend toont deze graduaatsproef aan dat automatische samenvatting van e-mails met behulp van artificiële intelligentie een haalbare en waardevolle technologie is binnen IT-supportcontexten. De gerealiseerde proof-of-concept vormt een solide uitgangspunt voor toekomstige integratie in bestaande systemen en verdere uitbreiding met bijkomende AI-functionaliteiten.

\selectlanguage{english}
}{}

%%---------- Samenvatting -----------------------------------------------------

\chapter*{\IfLanguageName{dutch}{Samenvatting}{Abstract}}

Deze graduaatsproef onderzoekt hoe automatische samenvatting van e-mails met behulp van artificiële intelligentie technisch gerealiseerd kan worden en welke meerwaarde dit kan bieden binnen IT-supportomgevingen. In veel servicedesk-omgevingen wordt e-mailverkeer gebruikt als basis voor verdere opvolging, maar deze communicatie is vaak lang, ongestructureerd en moeilijk snel te interpreteren.

Het doel van deze graduaatsproef is na te gaan hoe e-mails automatisch kunnen worden samengevat aan de hand van een AI-gestuurde oplossing en hoe deze samenvattingen kunnen worden weergegeven in een proof-of-concept ter ondersteuning van supportprocessen. De centrale onderzoeksvraag focust op de technische haalbaarheid, efficiëntie en kosteneffectiviteit van automatische e-mailsamenvatting met behulp van cloudgebaseerde AI.

De gekozen methodologie is praktijkgericht en richt zich op het ontwikkelen van een proof-of-concept. Hiervoor werd gebruikgemaakt van het GPT-4o-mini model via Azure AI Foundry. Een backend Web API werd ontwikkeld om e-mailinhoud te verwerken en door te sturen naar het AI-model. Gevoelige configuratiegegevens, zoals API-sleutels, worden veilig beheerd via user secrets. Daarnaast werd een eenvoudige frontend opgezet met React (Vite) om de volledige flow van e-mailinvoer tot gegenereerde samenvatting te demonstreren en te testen. De oplossing werd gevalideerd met behulp van Postman en end-to-end tests.

De resultaten tonen aan dat automatische e-mailsamenvatting technisch haalbaar is en consistente, bruikbare samenvattingen oplevert. Deze samenvattingen maken het mogelijk om sneller inzicht te krijgen in de kern van e-mailconversaties, wat een duidelijke potentiële meerwaarde biedt voor supportmedewerkers en ondersteunende processen.

De kostenanalyse wijst uit dat het gebruik van een cloudgebaseerd AI-model voor deze toepassing financieel haalbaar blijft, zelfs bij een relatief hoog aantal samenvattingen per dag. Hierdoor vormt de oplossing niet alleen een technisch, maar ook een economisch verantwoorde basis voor verdere ontwikkeling.

Samenvattend toont deze graduaatsproef aan dat automatische samenvatting van e-mails met behulp van artificiële intelligentie een haalbare en waardevolle technologie is binnen IT-supportcontexten. De gerealiseerde proof-of-concept vormt een solide uitgangspunt voor toekomstige integratie in bestaande systemen en verdere uitbreiding met bijkomende AI-functionaliteiten.
