%---------- Inleiding ---------------------------------------------------------

\section{Inleiding}%
\label{sec:inleiding}

In moderne supportomgevingen worden e-mails vaak automatisch gekoppeld aan tickets bij het aanmaken van nieuwe supportvragen. Deze e-mails bevatten doorgaans uitgebreide en ongestructureerde conversaties tussen klant en supportmedewerker. Het manueel doornemen en interpreteren van deze e-mailinhoud is tijdrovend en leidt regelmatig tot onvolledige of inconsistente documentatie binnen tickets.

De graduaatsproef situeert zich binnen de context van een IT-bedrijf dat gebruikmaakt van een interne supporttool waarin e-mails aan tickets worden gekoppeld. Supportmedewerkers zijn de primaire doelgroep van deze opdracht. Zij moeten snel inzicht krijgen in de kern van een probleem om efficiënt te kunnen reageren en oplossingen te formuleren.

De probleemstelling binnen deze casus is het ontbreken van een automatische en consistente manier om e-mailconversaties samen te vatten bij het aanmaken van tickets. Hierdoor gaat relevante informatie verloren en wordt de verwerkingstijd van tickets verhoogd. Dit leidt niet alleen tot een lagere efficiëntie binnen het supportteam, maar bemoeilijkt ook het hergebruik van tickets binnen een knowledge base.

De centrale onderzoeksvraag van deze graduaatsproef luidt als volgt:

\textit{Hoe kan een AI-gestuurde oplossing, gebaseerd op Azure OpenAI, ingezet worden om inkomende support-e-mails automatisch samen te vatten en hoe deze samenvatting in een proof-of-concept kan worden gegenereerd als voorbereiding op mogelijke integratie in een bestaande supporttool.}

De doelstelling van deze graduaatsproef is het uitwerken van een proof-of-concept die aantoont dat automatische e-mailsamenvatting met behulp van AI technisch haalbaar is en een meerwaarde biedt voor supportprocessen. Het concrete eindresultaat bestaat uit een werkende demo-oplossing met een backend en frontend die e-mails verwerkt en samenvat, aangevuld met een kosteninschatting en evaluatie van de oplossing.

%---------- Stand van zaken ---------------------------------------------------

\section{Stand van zaken}%
\label{sec:literatuurstudie}

De afgelopen jaren is het gebruik van artificiële intelligentie binnen IT-support sterk toegenomen. Vooral Natural Language Processing (NLP) speelt een belangrijke rol bij het analyseren en verwerken van ongestructureerde tekstuele data, zoals e-mails en chatberichten. Automatische tekstanalyse en samenvatting worden steeds vaker toegepast om grote hoeveelheden informatie sneller toegankelijk te maken.

Cloudplatformen zoals Microsoft Azure bieden gespecialiseerde AI-diensten aan waarmee ontwikkelaars zonder diepgaande kennis van machine learning toch geavanceerde toepassingen kunnen bouwen. Azure OpenAI maakt het mogelijk om krachtige taalmodellen, zoals GPT-4o-mini, te integreren via API-aanroepen. Deze modellen zijn in staat om context te begrijpen en beknopte samenvattingen te genereren op basis van lange tekstinvoer.

Binnen supportomgevingen bestaan reeds oplossingen voor ticketclassificatie en sentimentanalyse, maar automatische e-mailsamenvatting wordt vaak nog niet toegepast of volledig achterwege gelaten. Bestaande onderzoeken tonen aan dat automatische samenvatting kan bijdragen aan tijdsbesparing en betere kennisdeling, mits de output voldoende betrouwbaar en consistent is.

Het verschil met deze graduaatsproef is dat de focus ligt op een concrete bedrijfscontext en een praktische implementatie. In plaats van een theoretisch model wordt een toepasbare proof-of-concept ontwikkeld die gericht is op integratie in een bestaande supporttool.

%---------- Methodologie ------------------------------------------------------

\section{Methodologie}%
\label{sec:methodologie}

Deze graduaatsproef volgt een toegepaste onderzoeksaanpak waarbij een proof-of-concept centraal staat. Het onderzoek start met een analyse van de bestaande supportworkflow en de manier waarop e-mails momenteel worden verwerkt binnen tickets. Op basis hiervan worden functionele en technische vereisten opgesteld voor de AI-oplossing.

Vervolgens wordt Azure AI Foundry gebruikt om toegang te verkrijgen tot het GPT-4o-mini taalmodel via Azure OpenAI. Dit model wordt aangesproken via chat-completions om samenvattingen te genereren op basis van inkomende e-mailinhoud. De backend wordt uitgewerkt als een .NET Minimal API die verantwoordelijk is voor het ontvangen van e-mails, het verwerken van verschillende bestandsformaten (.txt, .eml en .msg) en het aanroepen van de Azure OpenAI API.

De frontend wordt ontwikkeld met React en dient als demo-interface waarin gebruikers e-mails kunnen uploaden of plakken. De volledige flow, van e-mailinvoer tot gegenereerde samenvatting, wordt getest en geëvalueerd. Daarnaast wordt aandacht besteed aan beveiliging, zoals het veilig opslaan van API-sleutels via user secrets.

Tot slot wordt een kostenberekening uitgevoerd op basis van het verwachte aantal samenvattingen per dag, maand en jaar. Deze berekening dient om de haalbaarheid van de oplossing binnen een bedrijfscontext te evalueren.

%---------- Verwachte resultaten ----------------------------------------------

\section{Verwacht resultaat en conclusie}%
\label{sec:verwachte_resultaten}

Het verwachte resultaat van deze graduaatsproef is een werkende proof-of-concept die aantoont dat automatische samenvatting van support-e-mails met behulp van Azure OpenAI technisch haalbaar is en een duidelijke meerwaarde biedt. De oplossing moet consistente en bruikbare samenvattingen genereren die geschikt zijn voor latere toevoeging aan tickets.

Voor de doelgroep, met name supportmedewerkers, betekent dit een snellere verwerking van tickets en een betere documentatie van klantproblemen. Daarnaast biedt de oplossing mogelijkheden voor verdere uitbreiding, zoals automatische ticketclassificatie of integratie met een knowledge base.

Hoewel de focus ligt op een proof-of-concept, vormt deze graduaatsproef een solide basis voor toekomstige integratie in een productieomgeving. Eventuele beperkingen of afwijkingen in de resultaten zullen worden geëvalueerd om aanbevelingen te formuleren voor verder onderzoek en optimalisatie.
